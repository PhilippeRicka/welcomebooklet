\documentclass[a5paper]{article}
\usepackage[utf8]{inputenc} 
\usepackage[T1]{fontenc}
\usepackage[french]{babel}
\usepackage{amsmath}
\usepackage{amsfonts}
\usepackage{amssymb}
\usepackage{amsthm}
\usepackage{graphicx}
\usepackage{lmodern}
\usepackage{microtype}
\usepackage{hyperref}
\usepackage{enumitem}

\title{Livret d'accueil}
\author{Alix Deleporte, Philippe Ricka}
\date{}

\newcommand{\todo}[1]{\textcolor{orange}{TODO:~ #1}~}

\begin{document}

\maketitle

\tableofcontents

\section{Bienvenue !}
\label{sec:bienvenue-}

\section{La vie de bureau}
\label{sec:la-vie-de}

\subsection{Bureaux et cobureaux}
\label{subsec:bureaux-et-cobureaux}

\subsection{Ordinateurs}
\label{subsec:ordinateurs}

Chaque doctorant a son ordinateur fixe attitré. Le service technique n'aime pas qu'on dérange les branchements, mais pas de panique, Alain Sartout\footnote{alain.sartout@math.unistra.fr} (bureau i108) et Alexis Palaticky\footnote{alexis.palaticky@math.unistra.fr} (bureau i103) sont disponibles pour tous nos soucis informatiques.


C'est également auprès d'Alexis qu'on peut emprunter du matériel comme les précieux adaptateurs pour vidéoprojecteurs.

\subsection{Imprimer}
\label{subsec:imprimer}

L'imprimante \verb pmath1  est accessible via les postes fixe sans restriction. Il est possible de configurer les ordinateurs personnels connectés au réseau \verb osiris-lab  pour imprimer à partir de ceux-ci. Toutefois, cette imprimante est réservée au petites impressions. Pour les gros articles et surtout pour les impressions relatives aux missions d'enseignement (énoncés de TD et d'examens, polys de cours) c'est l'imprimante du dépôt DALI à l'intersol. Elle nécessite des identifiants personnels à définir avec Daniel Grosson (\ref{sec:technique}) et permet 2000 impressions par an. Si cette limite est atteinte, il est possible de remettre les compteurs à zéro auprès de Daniel Grosson.

\subsection{Accès pass campus et autres problèmes techniques}
\label{subsec:technique}

Daniel Grosson\footnote{grosson@unistra.fr} et Philippe Sablon\footnote{philippe.sablon@unistra.fr} sont dans le local technique au rez-de-chaussée, après la salle C11. Ils s'occupent en particulier des droits d'accès aux bâtiments et au local à vélo via le pass campus.


Ils sont également chargés de toutes les questions matérielles comme les éventuels travaux et réparations. Pour les questions informatiques, voir la section \ref{sec:ordinateurs}.
\section{Manger}
\label{sec:manger}

\subsection{Restaurants universitaires}
\label{sec:rest-univ}

\subsection{Restaurant administratif}
\label{subsec:rest-admin}

\section{Séminaires et formations}
\label{sec:seminaires-et}

\subsection{Séminaires d'équipe}
\label{subsec:seminaires-dequipe}

Ils ont lieu théoriquement toutes les semaines au sein de chaque équipe. C'est l'occasion de rencontrer ses collègues Strasbourgeois ou non et de faire naître des collaborations et discussions intéressantes.

\subsection{Séminaire doctorant}
\label{subsec:seminaire-doctorant}

Le séminaire fait par des doctorants, pour des doctorants !

\vspace{1em}

Il est organisé par le président du séminaire et constitue l'occasion de se rencontrer, de se connaître, d'échanger autour de nos sujets respectifs et de nous entraider face aux difficultés inhérentes à notre statut. C'est également lors de ces séminaires que sont parfois discutées certaines questions d'envergure collective.

Traditionnellement, le séminaire a lieu tous les jeudis en fin d'après midi, est précédé d'un goûter (au frais de l'IRMA, mais c'est toujours chouette d'amener un gateau) et suivi d'un pot dans un bar des environs.

\vspace{1em}

L'organisation du séminaire des doctorants implique également celle d'un dîner de Noël.

\subsection{Séminaire d'École Doctorale}
\label{subsec:semin-decole-doct}

Les écoles doctorales organise régulièrement des séminaires. Notre école doctorale nous impose de suivre au moins 18 séminaires dont au moins 12 organisés par elle sur la totalité du doctorat. Il n'y a pas beaucoup de séminaires, il est donc conseillé de ne pas en rater.

\subsection{Formations doctorales}
\label{subsec:form-doct}

D'une manière générale, la plupart des formations est disponible sur la plateforme AMETHIS et nécessite une inscription régulière. Attention aux dates, certaines formations ne proposent qu'une très petite fenêtre pour l'inscription ! Celles-ci se décomposent en deux branches : les formations scientifiques et les formations transversales.


\vspace{1em}

Les modules scientifiques doivent représenter un volume horaire total de 40 à 50 heures sur tout le doctorat. Ils sont choisis parmi des formations scientifiques spécifiques, en général un cours de M2 dans nos domaines, un cours de M1 dans un autre domaine scientifique (physique, chimie, biologie, ...) ou une école d'été.

\vspace{1em}

Les formations transversales représentent un total de 20 à 25 heures sur tout le doctorat. Elles comprennent le module d'anglais obligatoire en deuxième année et sont censées nous donner des compétences non-scientifiques utiles à la poursuite de notre carrière.

\vspace{2em}

Ces informations se retrouvent sur le site de l'école doctorale : \href{http://ed.math-spi.unistra.fr/documents-et-liens-utiles/pour-les-formations-et-seminaires/}{http://ed.math-spi.unistra.fr/documents-et-liens-utiles/pour-les-formations-et-seminaires/}.

\subsection{Conférences et écoles d'été/d'hiver}
\label{subsec:conf-ecoledete}

Pour bien démarrer sa carrière de chercheur, le doctorant doit s'intégrer à la recherche nationale et internationale. Pour ce faire, il convient d'adopter une démarche en deux temps : connaître et faire connaître.

\subsubsection{\'Ecoles d'été/d'hiver}
\label{subsubsec:ecoledete}

Il est rare qu'un doctorant en première année ait des résultats à présenter devant ses pairs. Il vaut en général mieux assister à des conférences sans y exposer, mais participer à des écoles d'été ou d'hiver. Les écoles d'été ou d'hiver sont idéales pour connaître des chercheurs travaillant dans des domaines proches du nôtre, mais aussi et surtout pour actualiser nos connaissances dans ces domaines. C'est l'occasion de \textit{faire du réseau} et de rédiger un état de l'art, voire même d'entamer des collaborations avec d'autres chercheurs, doctorants ou non.

Les écoles permettent en général de publier un \textit{proceeding} qui récapitule le travail réalisé et le cas échéant les résultats qui en découlent.

\subsubsection{Conférences}
\label{subsubsec:conf}

Les conférences permettent d'accéder aux derniers résultats d'un domaine précis, mais permettent surtout de présenter ses résultats à ses pairs. C'est un outil de communication très prisé, et le choix de la conférence est important. Il convient d'avoir des résultats innovants à y montrer.

\section{Litige avec directeur ou encadrant}
\label{sec:litige}

\appendix

\section{Contacts}
\label{sec:contacts}




\end{document}