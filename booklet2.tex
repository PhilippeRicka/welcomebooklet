\documentclass[a5paper]{article}
\usepackage[utf8]{inputenc}
\usepackage[T1]{fontenc}
\usepackage[french]{babel}
\usepackage{amsmath}
\usepackage{amsfonts}
\usepackage{amssymb}
\usepackage{amsthm}
\usepackage{graphicx}
\usepackage{lmodern}
\usepackage{microtype}
\usepackage{hyperref}
\usepackage{enumitem}

\title{Livret d'accueil}
\author{Alix Deleporte, Philippe Ricka, Alexander Thomas}
\date{}

\newcommand{\todo}[1]{\textcolor{orange}{TODO:~ #1}~}

\begin{document}

\maketitle

\tableofcontents

\section{Bienvenue !}
\label{sec:bienvenue-}

Vous avez fait le choix hardi d'effectuer une thèse ! Bravo et bon courage !

Pour débuter bien dans ce monde nouveau et complexe, ce petit guide vous propose des explications et des informations comment obtenir de l'aide.

Penser à demander ou contacter d'autres doctorants en cas de nécessité ou de questions ! Ils ont probablement rencontré les mêmes difficultés que vous.


\subsection{Acteurs principaux}
\label{sec:acteurs}

Pendant votre thèse, il y aura trois acteurs principaux (en dehors de vous de vos collègues doctorants) :
\begin{itemize}

\item Le directeur de thèse
\item L'Ecole Doctorale (ED)
\item L'université

\end{itemize}

Le rôle du directeur de thèse (voir les détails dans la section \ref{direction-these} consiste principalement à vous conseiller et suivre au niveau de votre projet mathématique. Il surveille votre avancement, veille au bon déroulement de la thèse et répond à vos questions de maths et sur le monde académique.

L'ED est l'organe qui administre le déroulement de la thèse, de l'inscription à la soutenance. Tous les ans, vous devez vous (ré-)inscrire à l'ED. En tant que doctorant vous devez suivre des formations et des séminaires validés par l'ED (voir les détails dans les sections \ref{semin-decole-doct} et \ref{form-doct}).

L'université de Strasbourg vous délivrera le diplôme de thèse après votre soutenance. Pendant toute la thèse vous devez vous (ré-)inscrire tous les ans à l'université ce qui vous permet d'avoir le statut d'étudiant. C'est aussi l'université avec qui vous signez votre contrat doctoral et qui vous paie. Dans le cadre de votre thèse vous pouvez effectuer une mission d'enseignement, voir \ref{mission-enseignement}.


\section{La vie de bureau}
\label{sec:la-vie-de}


\subsection{Bureaux et cobureaux}
\label{subsec:bureaux-et-cobureaux}

La secrétaire de l'UFR (Mme Igot) vous attribue un bureau. Ce sera votre espace de travail principal pour les années à venir !
La plupart des bureaux se trouvent au premier étage, mais il y en a aussi dans tous les autres (avec une décroissance exponentielle).

Installez-vous confortablement dans ce lieu et n'hésitez pas à changer la disposition des tables (avec l'accord de vos cobureaux bien sûr !), à ajouter des posters sur les murs (pas sur la porte), à acheter une machine à café (s'il n'y en a pas), à nettoyer et à jeter de vieux documents de vos prédécesseurs (vérifier bien auprès de vos cobureaux que ce sont des documents inutiles !).

\subsection{Ordinateurs}
\label{subsec:ordinateurs}

Chaque personne travaillant au laboratoire a son ordinateur fixe attitré. Le service technique n'aime pas qu'on dérange les branchements, mais pas de panique, Alain Sartout\footnote{alain.sartout@math.unistra.fr} (bureau i108) et Alexis Palaticky\footnote{alexis.palaticky@math.unistra.fr} (bureau i103) sont disponibles pour tous nos soucis informatiques.


C'est également auprès d'Alexis qu'on peut emprunter du matériel comme les précieux adaptateurs pour vidéoprojecteurs.

\subsection{Imprimer}
\label{subsec:imprimer}

Les imprimantes \verb pmath1 à \verb pmath4 (en fonction de votre
étage)  sont accessibles via les postes fixes sans restriction. Il est
possible de configurer les ordinateurs personnels connectés au réseau
\verb osiris-lab  pour imprimer à partir de ceux-ci. Toutefois, ces
imprimantes sont réservées au petites impressions. Pour les gros
articles et surtout pour les impressions relatives aux missions
d'enseignement (énoncés de TD et d'examens, polys de cours), il faut utiliser l'imprimante du dépôt DALI à l'intersol. Elle nécessite des identifiants personnels à définir avec Daniel Grosson (\ref{sec:technique}) et permet 2000 impressions par an. Si cette limite est atteinte, il est possible de remettre les compteurs à zéro auprès de Daniel Grosson.

\subsection{Accès pass campus et autres problèmes techniques}
\label{subsec:technique}

Daniel Grosson\footnote{grosson@unistra.fr} et Philippe
Sablon\footnote{philippe.sablon@unistra.fr} sont dans le local
technique au rez-de-chaussée de l'UFR, après la salle C11. Ils s'occupent en particulier des droits d'accès aux bâtiments et au local à vélo via le pass campus.


Ils sont également chargés de toutes les questions matérielles comme
les éventuels travaux et réparations. Pour les questions
informatiques, voir la section \ref{sec:ordinateurs}.

\subsection{Outils numériques}
\label{sec:outils-numeriques}
Tout le monde a déjà fait l'expérience d'un fichier perdu à cause
d'une clé USB défaillante, d'une mauvaise manipulation ou d'une mise à
jour fatidique. Il est important de sauvegarder vos travaux à
plusieurs endroits !

\begin{itemize}
\item En plus de l'espace disque de poste fixe, vous disposez d'un
  espace sur un serveur de fichiers à l'IRMA. Celui-ci s'appelle
  irma-file. Les données qui y sont enregistrées sont régulièrement
  journalisées et peuvent être récupérées en cas de pépin. Mettez-y
  vos fichiers régulièrement.
\item Vous
  pouvez utiliser le serveur Owncloud mis à votre
  disposition par le laboratoire. Owncloud fonctionne comme d'autres serveurs de fichiers
  partagés : après avoir installé un logiciel sur votre machine, un
  dossier particulier sera synchronisé en continu avec le
  serveur. Vous pouvez ainsi synchroniser vos fichiers entre plusieurs
  ordinateurs en toute transparence. Depuis l'interface web d'Owncloud, il est possible de
  partager un dossier avec un autre membre du labo.
\item L'université met à votre disposition un serveur Seafile, qui
  fonctionne de la même manière.
\item Le laboratoire possède un
  serveur gitlab sur lequel vous pouvez collaborer avec git sur des
  projets en \LaTeX~ou autres codes sources.
\end{itemize}


\subsection{Bibliothèque}
\label{sec:biblio}

La fac de maths a sa propre bibliothèque de maths, situé au rez-de-chaussée. On y trouve un espace pour étudiants avec des ouvrages pour Licence jusqu'au Master et l'Agrégation mais aussi un espace pour chercheurs avec des ouvrages spécialisés ainsi que les revues et journaux scientifiques. Pour accéder à l'espace des chercheurs et pour emprunter des ouvrages, il faut vous faire une carte auprès de M Thureau ou un des personnels. Cette carte permet également l'accès à la bibliothèque en dehors des horaires d'ouverture.
Vous pouvez emprunter des ouvrages pendant 1 mois, puis les prolonger tant que personne ne pose une option sur l'ouvrage.


\section{Manger}
\label{sec:manger}

\subsection{Restaurants universitaires}
\label{sec:rest-univ}
Puisque vous êtes doctorant.e, vous êtes étudiant.e ! Une fois que
vous aurez reçu votre carte étudiante, vous pourrez
aller manger aux différents restaurants universitaires pour une somme
modique. Les restaurants les plus proches sont :
\begin{itemize}
\item Esplanade ;
\item Paul Appell ;
\item Gallia
\item FEC.
\end{itemize}

La plupart des restos U sont également ouverts le soir et les week-ends. Renseignez-vous sur le site du CROUS : http://www.crous-strasbourg.fr/.


\subsection{Restaurant administratif}
\label{subsec:rest-admin}
Puisque vous êtes doctorant.e, vous êtes chercheur.e ! Vous pouvez
faire une demande de carte du restaurant administratif qui vous
permettra d'aller manger avec tous les autres membres de votre équipe.

\section{Séminaires et formations}
\label{sec:seminaires-et}

\subsection{Séminaires d'équipe}
\label{subsec:seminaires-dequipe}

Ils ont lieu théoriquement toutes les semaines au sein de chaque équipe. C'est l'occasion de rencontrer ses collègues Strasbourgeois ou non et de faire naître des collaborations et discussions intéressantes.

\subsection{Séminaire doctorant}
\label{subsec:seminaire-doctorant}

Le séminaire fait par des doctorant.e.s, pour des doctorant.e.s !

\vspace{1em}

Il est organisé par le président du séminaire et constitue l'occasion de se rencontrer, de se connaître, d'échanger autour de nos sujets respectifs et de nous entraider face aux difficultés inhérentes à notre statut. C'est également lors de ces séminaires que sont parfois discutées certaines questions d'envergure collective.

Traditionnellement, le séminaire a lieu tous les jeudis en fin d'après midi, est précédé d'un goûter (au frais de l'IRMA, mais c'est toujours chouette d'amener un gateau) et suivi d'un pot dans un bar des environs.

\vspace{1em}

L'organisation du séminaire doctorant implique également celle d'un dîner de Noël et d'un barbecue de fin d'année universitaire.

\subsection{Séminaire d'École Doctorale}
\label{subsec:semin-decole-doct}

Les écoles doctorales organise régulièrement des séminaires. Notre école doctorale nous impose de suivre au moins 18 séminaires dont au moins 12 organisés par elle sur la totalité du doctorat. Il n'y a pas beaucoup de séminaires, il est donc conseillé de ne pas en rater.

Depuis la rentrée 2018 les colloquiums, organisé par l'IRMA à peu près une fois par mois autour d'un mathématicien ou d'une mathématicienne de renommée, comptent parmi les séminaires ED.

\subsection{Formations doctorales}
\label{subsec:form-doct}

D'une manière générale, la plupart des formations est disponible sur la plateforme AMETHIS et nécessite une inscription régulière. Attention aux dates, certaines formations ne proposent qu'une très petite fenêtre pour l'inscription ! Celles-ci se décomposent en deux branches : les formations scientifiques et les formations transversales.


\medskip

Les modules scientifiques doivent représenter un volume horaire total de 40 à 50 heures sur tout le doctorat. Ils sont choisis parmi des formations scientifiques spécifiques, en général un cours de M2 dans nos domaines, un cours de M1 dans un autre domaine scientifique (physique, chimie, biologie, ...) ou une école d'été.

\medskip

Les formations transversales représentent un total de 20 à 25 heures sur tout le doctorat. Elles comprennent le module d'anglais obligatoire en deuxième année et sont censées nous donner des compétences non-scientifiques utiles à la poursuite de notre carrière.

\medskip

Ces informations se retrouvent sur le site de l'école doctorale : \href{http://ed.math-spi.unistra.fr/documents-et-liens-utiles/pour-les-formations-et-seminaires/}{http://ed.math-spi.unistra.fr/documents-et-liens-utiles/pour-les-formations-et-seminaires/}.

\subsection{Conférences et écoles d'été/d'hiver}
\label{subsec:conf-ecoledete}

Pour bien démarrer votre carrière, vous devez vous intégrer à la
communauté universitaire nationale et internationale. Pour ce faire, il convient d'adopter une démarche en deux temps : connaître et faire connaître.

\subsubsection{\'Ecoles d'été/d'hiver}
\label{subsubsec:ecoledete}

Les écoles d'été ou d'hiver permettent une introduction à un domaine
et sont destinées aux jeunes doctorant.e.s. Elles sont idéales pour connaître des chercheurs travaillant dans des domaines proches du vôtre, mais aussi et surtout pour actualiser vos connaissances dans ces domaines. C'est l'occasion de \textit{faire du réseau} et de rédiger un état de l'art, voire même d'entamer des collaborations avec d'autres chercheurs, doctorants ou non.

Les écoles permettent parfois de publier un \textit{proceeding} qui récapitule le travail réalisé et le cas échéant les résultats qui en découlent.

\subsubsection{Conférences}
\label{subsubsec:conf}

Les conférences permettent d'accéder aux derniers résultats d'un
domaine précis, mais permettent surtout de présenter ses résultats à
ses pairs. C'est un outil de communication très prisé, et le choix de
la conférence est important.


\section{Mission d'enseignement}

Dans le cadre de votre thèse, vous pouvez, d'une façon facultative, effectuer une mission d'enseignement à l'université de Strasbourg. Cela vous permet d'acquérir de l'expérience dans l'enseignement, de valider le stage de l'agrégation (si vous l'avez passée) et de gagner tout simplement un peu plus d'argent.

Une mission d'enseignement est de 64h pour l'année universitaire (ce qui revient à environ 2h par semaine) et peut être divisée en plusieurs sous-missions. Par exemple vous avez un TD de 32h et 2h de khôlle en MPA (parcours maths approfondi à la fac) au premier semestre et un cours intégré de 24h au deuxième semestre. Les cours intégrés comptent avec un coefficient de 1,25 (car nécessite un temps de préparation plus important), ainsi les 24h de cours intégré comptent pour 30h, ce qui fait un total de 64h.

Vous n'avez pas le droit de donner des khôlles en classe préparatoire.

Les missions d'enseignement sont distribuées par un chercheur (actuellement Emanuel Opshtein) qui fait en sorte que chacun ait bien 64h. Vous pouvez faire des voeux si vous préférez enseigner tel ou tel cours.

Il est également possible d'effectuer une mission de moins de 64h dans le cadre d'une vacation. Vous serez payé non pas tous les mois mais seulement deux ou trois fois par an, souvent avec un retard de plusieurs mois.


\section{La direction de thèse}
\label{sec:direction-these}

\subsection{Responsabilités du directeur}

lancement du projet, idées, pistes

surveillance de l'avancement

bon déroulement (en particulier veille à ce que personne d'autre ne fasse la même rechercher que vous, ou bien vous indique des collaborations possibles)

questions de contenu, de rédaction, concernant le monde de la recherche

rendez-vous (une fois par semaine...)

faire connaissance avec d'autres experts de votre domaine, indication d'école d'été ou de conférences à participer

articles scientifiques (rédaction, revue/journal)

soutenance


\subsection{Grandes lignes de la thèse}

MOOC

Présentation d'un poster en début de 2e année

Cours d'anglais obligatoire au milieu de la 2e année

Présentation de mi-thèse à la fin de la 2e année devant un jury (puis à la fin de chaque année suivante)




\appendix


\section{Contacts}
\label{sec:contacts}

Secrétaires IRMA
Secrétaire UFR (M Igot)
Informatique (M Sartout et M Palaticky)
M Grosson et M Sablon

ED (Mme Kostmann)
DRH (fiches de paye)
Inscription administrative (Mme Lapierre)



\end{document}
